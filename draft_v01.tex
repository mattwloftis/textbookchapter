\PassOptionsToPackage{unicode=true}{hyperref} % options for packages loaded elsewhere
\PassOptionsToPackage{hyphens}{url}
%
\documentclass[12pt,]{article}
\usepackage[]{cochineal}
\usepackage{amssymb,amsmath}
\usepackage{ifxetex,ifluatex}
\usepackage{fixltx2e} % provides \textsubscript
\ifnum 0\ifxetex 1\fi\ifluatex 1\fi=0 % if pdftex
  \usepackage[T1]{fontenc}
  \usepackage[utf8]{inputenc}
  \usepackage{textcomp} % provides euro and other symbols
\else % if luatex or xelatex
  \usepackage{unicode-math}
  \defaultfontfeatures{Ligatures=TeX,Scale=MatchLowercase}
\fi
% use upquote if available, for straight quotes in verbatim environments
\IfFileExists{upquote.sty}{\usepackage{upquote}}{}
% use microtype if available
\IfFileExists{microtype.sty}{%
\usepackage[]{microtype}
\UseMicrotypeSet[protrusion]{basicmath} % disable protrusion for tt fonts
}{}
\IfFileExists{parskip.sty}{%
\usepackage{parskip}
}{% else
\setlength{\parindent}{0pt}
\setlength{\parskip}{6pt plus 2pt minus 1pt}
}
\usepackage{hyperref}
\hypersetup{
            pdftitle={`Big data' og politologisk datavidenskab},
            pdfauthor={Frederik Hjorth; Matt W. Loftis},
            pdfborder={0 0 0},
            breaklinks=true}
\urlstyle{same}  % don't use monospace font for urls
\usepackage[margin=1in]{geometry}
\usepackage{graphicx,grffile}
\makeatletter
\def\maxwidth{\ifdim\Gin@nat@width>\linewidth\linewidth\else\Gin@nat@width\fi}
\def\maxheight{\ifdim\Gin@nat@height>\textheight\textheight\else\Gin@nat@height\fi}
\makeatother
% Scale images if necessary, so that they will not overflow the page
% margins by default, and it is still possible to overwrite the defaults
% using explicit options in \includegraphics[width, height, ...]{}
\setkeys{Gin}{width=\maxwidth,height=\maxheight,keepaspectratio}
\setlength{\emergencystretch}{3em}  % prevent overfull lines
\providecommand{\tightlist}{%
  \setlength{\itemsep}{0pt}\setlength{\parskip}{0pt}}
\setcounter{secnumdepth}{5}
% Redefines (sub)paragraphs to behave more like sections
\ifx\paragraph\undefined\else
\let\oldparagraph\paragraph
\renewcommand{\paragraph}[1]{\oldparagraph{#1}\mbox{}}
\fi
\ifx\subparagraph\undefined\else
\let\oldsubparagraph\subparagraph
\renewcommand{\subparagraph}[1]{\oldsubparagraph{#1}\mbox{}}
\fi

% set default figure placement to htbp
\makeatletter
\def\fps@figure{htbp}
\makeatother

\usepackage{etoolbox}
\makeatletter
\providecommand{\subtitle}[1]{% add subtitle to \maketitle
  \apptocmd{\@title}{\par {\large #1 \par}}{}{}
}
\makeatother
\usepackage[]{natbib}
\bibliographystyle{apsr}

\title{`Big data' og politologisk datavidenskab}
\providecommand{\subtitle}[1]{}
\subtitle{Udkast, april 2020}
\author{Frederik Hjorth \and Matt W. Loftis}
\date{}

\begin{document}
\maketitle

`Big data' er overalt. Det gælder i dobbelt forstand: takket være
drastiske stigninger i computeres hukommelse og regnekraft indeholder
næsten alle computere i dag store, ustrukturerede datamængder. Mange af
disse data er biprodukter af menneskelig adfærd, som i dag registreres
og kvantificeres i historisk uset omfang. Men big data er også overalt i
den forstand at begrebet `big data' og beslægtede begreber er blevet
almindeligt kendte og bredt anvendte, og ikke mindst genstand for stor
kommerciel interesse. En hyppigt citeret artikel fra \emph{Harvard
Business Review} kaldte således ``data scientist'' for ``det 21.
århundredes mest sexede job'' \citep{davenport2012data}.

Alene den begrebslige udbredelse af big data gør det relevant at vide
hvad det nærmere dækker over. Men big data er også reelt et væsentligt
nybrud i forhold til de data og metoder, politologi og samfundsvidenskab
traditionelt har betjent sig af. Big data muliggør analyser af
politologiske emner som ville have været umulige med traditionelle
metoder, men kræver også nye teknikker og metodiske værktøjer.

Formålet med dette kapitel er at introducere til de datatyper og
metoder, begrebet big data dækker over. Først opridser vi begrebets
betydning og historie. Dernæst diskuterer vi hvordan en række
karakteristika ved big data skaber særlige udfordringer i forhold til at
udvikle stærke forskningsdesigns. Herefter præsenterer vi en række
specifikke tekniske værktøjer til behandling af big data. Afslutningsvis
opridser vi nogle væsentlige etiske problematikker i relation til brugen
af big data.

cites: \citet{mullainathan2017machine}, \citet{varian2014big}

\hypertarget{hvad-er-big-data}{%
\section{Hvad er big data?}\label{hvad-er-big-data}}

I en toneangivende artikel peger \citet{lazeretal} på big data som
kilden til en ny type samfundsvidenskab, ``computational social
science'', med ``kapacitet til at indsamle og analysere data med
historisk uset bredde, dybde og omfang''. Men begrebet big data lever to
liv. På den ene side er den populære definition af begrebet forbundet
med futuristiske løfter om ny, datadrevet videnskab og teknologi. På den
anden side står hvad man kunne kalde den operationelle definition af
hvordan store datamænger indsamles, lagres og analyseres af
samfundsvidenskabsfolk. For at belyse betydningen af big data for
politologi betragter vi først denne anden, operationelle betydning af
begrebet inden vi vender tilbage til den første, populære betydning.

En bredt anvendt operationel definition identificerer big data med de
såkaldte `tre V'er': \emph{Volume}, \emph{Variety} og \emph{Velocity},
dvs. omfang, variation og hastighed \citep{laney01}. Big data refererer
her til foranderlige og meget store datasæt -- inklusive data der er for
store til at gemme på en almindelig pc -- der indeholder masser af
variation. Datakilder af denne art er ofte interessante for
samfundsvidenskab: søgemaskinelogfiler, sociale medieaktiviteter,
offentlige registre, mobiltelefonregistre eller endda data gemt ved
passiv overvågning udført af digitale enheder i den fysiske verden er
alle blevet anvendt til samfundsvidenskabelig forskning
\citep{salganik17}. Adgang til disse data kræver partnerskaber med deres
ejere -- telefonfirmaer, regeringer, teknologiselskaber osv. Dette kan
indebære at man skriver software til at få adgang til websteder eller
eksterne databaser eller kan involvere mere formaliserede partnerskaber
for at dele information sikkert \citep{EL14}. I afsnittet om
``Anskaffelse af data'' nedenfor beskriver vi nærmere hvordan det kan
finde sted.

Dette peger også på en vigtig forskel mellem big data og traditionelle
samfundsvidenskabelige data: traditionelle samfundsvidenskabelige data
er typisk indsamlet med samfundsvidenskab som formål. I modsætning
hertil omtales big data til tider som ``fundne data'' eller ``digital
udstødning'' \citep{harford14}. Hvad betyder det? Næsten alle big data
er udviklet til \emph{andre formål} end samfundsforskning. Metadata
såsom tidsmarkører, antal følgere, eller aktivitetsmål på sociale medier
lagres ikke for videnskabens skyld - eller nødvendigvis i
overensstemmelse med videnskabelige standarder.

Selv offentlige datakilder kan udvise dette problem. Betragt for
eksempel et lille, men illustrativt eksempel: lov nr. 1049 af
11/12/1996. Loven er ganske kort, færre end 20 ord. Det eneste den gør
er at ophæve lov om skoleskibsafgift. Civilstyrelsens database med al
dansk lovgivning, Retsinformation, angiver adskillige stykker metadata
om lov nr. 1049 (jf.
\texttt{https://www.retsinformation.dk/Forms/R0710.aspx?id=83509}).
Metadata angiver f.eks. den lov der ophæves, adskillige relaterede
dokumenter, lovens offentliggørelsesdato, og dens ministerområde. Men
dette sidste datapunkt er lidt forvirrende. Ministerområdet for lov nr.
1049 er angivet som ``Uddannelses- og Forskningsministeriet''. Men loven
er underskrevet af Mimi Jakobsen, som var erhvervsminister i regeringen
Poul Nyrup Rasmussen II. Så hvorfor er loven ikke tilknyttet
Erhvervsministeriet? Fejlen opstår fordi Civilstyrelsen opdaterer
lovgivning løbende så den afspejler lovgivningens ressortområde \emph{i
dag}. Koblingen af lov nr. 1049 til Uddannelses- og
Forskningsministeriet er indlysende forkert hvis du skal bruge
historiske data om dansk lovgivning. Uheldigvis for politologer er det
korrekt hvis du - som Civilstyrelsen - ikke har til formål at bedrive
historisk forskning, men i stedet vil organisere gældende dansk
lovgivning efter ministerområder. For at gøre ondt værre kan
Retsinformation ændre sig yderligere i fremtiden på måder der ikke
gavner politologien, alt efter hvad der tjener Civilstyrelsens behov.

Når man analyserer big data der er produceret til et eksisterende privat
eller offentligt formål lurer problemer af denne type konstant. Selvom
big data kan være enormt værdifulde for samfundsvidenskaben er deres
værdi en \emph{utilsigtet bivirkning} af kommerciel aktivitet eller
myndighedsudøvelse. Når vi anvender big data i forskningsøjemed er det
derfor altid vigtigt at forstå hvorfor og hvordan data er opstået til at
begynde med, og tænke igennem hvilke implikationer det har for vores
videnskabelige anvendelse. Som \citet{salganik17} formulerer det følger
både udfordringer og muligheder ved big data af at spørge sig selv
hvorfor data blev indsamlet i første omgang.

\hypertarget{kilder-til-big-data}{%
\section{Kilder til big data}\label{kilder-til-big-data}}

Det er ofte en møjsommelig proces at indsamle samfundsvidenskabelige
data. Derfor fremhæves det ofte som en fordel ved big data at
undersøgelsens subjekter selv genererer data: en forsker kan eksempelvis
indsamle millioner af tweets om et politisk emne uden skulle uddele et
eneste spørgeskema. Men selv om data er genereret på forhånd er det ikke
ligetil at \emph{anskaffe} sig data. Der er groft sagt tre måder man kan
gøre det på.

Den første og mest umiddelbare måde er at udtrække data direkte fra
websider, typisk kaldet \emph{scraping}. Scraping udnytter at indholdet
på de fleste større websider kommer fra databaser som fremstiller
indholdet i websider med en konsistent struktur. Ved at hente kildekoden
til disse websider, på samme måde som en webbrowser gør det, kan man
udtrække data på en konsistent måde. Hvis man f.eks. besøger websiden
for \emph{Lov om ophævelse af lov om skoleskibsafgift} hos
Retsinformation finder man i sidens kildekode bl.a. dette:

\begin{verbatim}
<div class="metadata-summary">
            <span class="kortNavn">LOV nr 1049 af 11/12/1996 Gældende</span><br>
      <div class="ressort">
                Offentliggørelsesdato: 12-12-1996<br>
        Uddannelses- og Forskningsministeriet
            </div>
        </div>
\end{verbatim}

Kodestumpen viser at Retsinformations database lagrer lovens navn i
feltet \texttt{kortNavn} og lovens offentliggørelsesdato og
ressortområde i feltet \texttt{ressort}. Takket være den stringente
kodestruktur er det nemt at gemme disse og andre metadata i et
analyserbart format. Og fordi kodestrukturen er ens på tværs af love hos
Retsinformation kan man scrape data om tusindvis af andre love med samme
lille stykke kode.

Når man indsamler data ved hjælp af scraping tilgår man i princippet
data på samme måde som en almindelig internetbruger der benytter sig af
en browser. Men fordi scraping gør det muligt at hente kolossale
datamængder er det også en kontroversiel praksis. Et illustrativt
eksempel på det kommer fra en meget omtalt juridisk strid mellem det
sociale netværk LinkedIn og analysefirmaet HiQ. En del af HiQ's
forretningsmodel er at analysere arbejdsmarkedet for it-specialister, og
HiQ har bl.a. høstet data ved at scrape data fra offentlige profiler på
LinkedIn. I 2017 sagsøgte LinkedIn HiQ med påstand om at HiQ's
scraping-praksis var et brud på amerikansk it-lovgivning. HiQ fik til
sidst medhold i at virksomheden kunne scrape data fra offentlige
LinkedIn-sider uden tilsagn fra LinkedIn, men sagen illustrerer at
scraping ofte finder sted i en juridisk gråzone.

Kodestumpen om \emph{Lov om ophævelse af lov om skoleskibsafgift} kommer
fra Retsinformation, og det er som hovedregel ikke forbudt at scrape
data fra offentlige hjemmesider, så længe man ikke urimeligt belaster
udbyderens servere. Man bør dog uanset kilden altid sikre sig tilsagn
fra dataudbyderen før man går i gang med at scrape data.

En anden måde at hente data på er gennem såkaldte API'er. API står for
\emph{Application Programming Interface} og er en slags kontrolleret
adgang til data hos en dataudbyder. API'er indebærer altså ikke samme
juridiske usikkerheder som scraping, da udbyderen selv stiller data til
rådighed og definerer rammerne herfor. Eksempelvis har mange API'er
\emph{rate limits} der sætter grænser for hvor meget data man kan hente
ad gangen.

Princippet om at big data ikke er lavet for samfundsforskningens skyld
gælder også for API'er. Det egentlige formål for de fleste API'er er at
dele data på tværs af kommercielle platforme. For eksempel er det API'er
der muliggør at et online-medie kan vise hvilke af ens egne
Facebook-venner der har `liket' en specifik artikel, fordi avisen kan
tilgå data om læserens Facebook-netværk gennem Facebooks API. Men mange
sociale netværk stiller meget righoldige data til rådighed for forskere
gennem API'er. For eksempel bruger \citet{hjorth2019}, som studerer
rækkevidden af online misinformation, Twitters API til at indsamle data
om ca. 13 millioner følgere af ca. 10.000 Twitter-konti. Offentlige
myndigheder stiller også i stigende grad data til rådighed gennem
API'er. Eksempelvis stiller Folketinget data om medlemmer, forhandlinger
og lovarbejde til rådighed gennem en API.

En tredje måde at få adgang til big data er gennem et egentligt
samarbejde med virksomheder der lagrer big data. For eksempel
rapporterer \citet{bond2012} om et eksperiment, hvor samfundsforskere i
samarbejde med Facebook randomiserede hvilken type information
Facebook-brugere fik om deres venners stemmeadfærd. I kraft af
samarbejdet kunne forskerne udføre eksperimentet i en uhørt stor skala:
eksperimentet involverede i alt 61 millioner Facebook-brugere.

Studiet af Bond m.fl. er exceptionelt fordi det kombinerer kvaliteterne
ved big data og eksperimentel metode. Mange forskere gør derfor også en
stor indsats for at etablere samarbejder med virksomheder og
organisationer der kan give dem adgang til data, der ellers ville være
utilgængelige. Men samarbejde med virksomheder om big data er ikke uden
faldgruber. For det første kræver det ofte et betydeligt bureaukratisk
benarbejde at etablere et samarbejde. For det andet, og mere principielt
problematisk, er virksomheder og organisationer sjældent interesserede i
forskning der stiller dem selv i et dårligt lys. Det kan betyde at nogle
typer undersøgelser prioriteres på bekostning af andre, alene fordi de
passer bedre til store teknologivirksomheders dagsordener. Eksempelvis
konkluderede \citet{bond2012} at Facebook-kampagnen havde en gunstig
effekt på valgdeltagelse. Det er i sagens natur en flatterende
konklusion for Facebook. Men det er uklart om forskerne havde haft samme
frihedsgrader til at studere de negative konsekvenser af at bruge
Facebook.

\hypertarget{big-data-og-forskningsdesign}{%
\section{Big data og
forskningsdesign}\label{big-data-og-forskningsdesign}}

Big data has applications in political science in empirical studies of
all types, from description to explanation, prediction, and causal
studies. Although these applications are only beginning across the
social sciences, the past 15 years have provided enough experience that
we can already point to one strong finding that can always guide us when
we apply big data in our work: research design still matters
\citetext{\citealp[see][p.~13]{toshkov16}; \citealp{CG15}}. Here we
discuss aspects of research design that deserve special attention when
working with big data, namely, biases in the data and the definition of
the unit of analysis.

\emph{Familiar sampling bias}

Perhaps the most heady idea about big data was expressed best in a, now
infamous, article in \emph{Wired} magazine by Chris \citet{anderson08}
-- the idea that big data eliminates sampling problems because
\emph{n}=all. That is, one can analyze \emph{all of the data}. The
article has met with a good deal of attention and pushback in the years
since it was published. In early 2020, it boasts more than 2.000
citations on Google Scholar. To put that in perspective, Maurice
Duverger's famous book \emph{Political Parties}, the origin of the
eponymous Duverger's Law, was published in 1959 and counts just over
7.000 citations.

As a concept, \emph{n}=all implies that big data can be taken at face
value. Its patterns reveal a complete picture of human behavior.
Unfortunately, the selection problem is still with us. It arises
whenever observations enter your data for reasons that systematically
relate to the outcome variable. \citet{AP08} illustrate the problem with
an example from the United States' National Health Interview Survey, in
which individuals are asked to rate their health on a scale. Perhaps
unsurprisingly, they find that individuals in hospitals consistently
rate their health as worse than those out of hospital. Then they pose an
interesting question: should we conclude that hospitals make people
sicker? The answer is, of course, no. We know that because we already
understand that individuals in hospital \emph{selected} into being there
precisely because they were sick. Failing to recognize that would result
in sampling bias.

Similar situations regularly arise with big data. For example,
\citet{harford14} recounts the story of Boston's experiment with using
cell phone tracking to identify pot holes in the city's streets that
need repair. The result was that the city discovered every pothole in
neighborhoods frequented by young, affluent drivers--the profile of the
type of person who owned a smart phone and downloaded the city's app.
Uncovering a similar problem, \citet{TSSW10} confirm a finding in
previous research that Twitter mentions of German parties correlated
strongly with their vote share in the 2009 parliamentary elections, with
one major exception. The extremely online \emph{Pirate Party} garned a
huge number of mentions on Twitter while receiving a tiny fraction of
the actual vote. In both of these cases, if \emph{n}=all then it equals
``all'' of a specific, self-selected group of technology users.

Ignoring sampling bias has always meant that researchers risked finding
the wrong answer. Using big data with biased samples simply means now we
get extremely precise estimates of the wrong answer.

\emph{Special headaches for big data}

Big data confronts us with new sources of bias and confounding. Perhaps
the most famous example for social scientists is that of Google Flu
Trends (GFT). GFT was a Google project launched in 2008 that predicted
regional flu epidemics from fine-grained data on users' Google queries
about likely flu symptoms \citep{Getal2009}. After initially receiving
attention for its impressive accuracy, GFT's predictive performance
began to diminish over time until--by the time the project was shuttered
in 2015--for years it had produced inaccurately high forecasts5
sometimes as great as double the number of flu reported by the U.S.
Centers for Disease Control and Prevention \citep{harford14}. The
reasons for GFT's collapse are instructive for understanding both why
research design still matters and what new headaches come with big data.

Several things contributed to GFT's problems, but the most noteworthy
seems to have arisen after Google adjusted its main search service
\citep[see][]{LKKV14}. Like any organization that produces or processes
big data, Google changes its service or its algorithms from time to
time. In 2011 and 2012, Google added functions to its search service
that suggested additional search terms to users, based on their original
search terms. It even suggested possible diagnoses for search terms that
involved disease symptoms. The upshot was an apparent feedback loop:
users of Google search were nudged to refine their searches for symptoms
and to seek out information on the flu, intensifying the signals that
GFT relied on to predict flu outbreaks. There is more to GFT's story, of
course, but this part of it points out some special features of big data
researchers must consider to avoid sampling bias. As \citet{salganik17}
puts it, big data is \emph{algorithmically confounded}, \emph{dirty},
and \emph{drifting}.

Algorithmic confounding happens when computer behavior interacts with
human behavior in a system, altering the human behavior we want to
study. GFT is an example, since computer behavior--the Google search
algorithm's recommendations--interacted with the human behavior that GFT
relied on to predict flu outbreaks (searches for flu symptoms). Google's
search suggestions altered users' perceptions of their symptoms and led
them to different search behavior.

By ``dirty,'' \citet{salganik17} means that big data also contains false
positives in the form of computer behavior that is mistaken for human
behavior. For example, at least a portion of Google's additional search
traffic around flu syptoms may have been purely accidental, caused only
by the search engine's suggestions. Smart phone users will recognize
this problem as an autocorrect fail. An example that occassionaly makes
headlines would be the presence of large numbers of ``bots'' on social
media platforms. Bots are accounts that are not operated by human users,
but rather are simply software programs executing automated behaviors in
the form of liking, following, or commenting. Despite Twitter and other
platforms' occasional purges of bot accounts, an unspecified and
probably high number of bots persist. Any analysis of social media data
must think carefully about how to clean the data to eliminate as much
bot behavior as possible.

Drift refers to the occasional changes that occur in how big data are
collected, a problem we foreshadowed in the first section. GFT was a
victim of the drift in Google's search data that occurred after the
introduction of search suggestions in 2011. Fundamentally, GFT's data
before and after 2011 came from different populations--one in which
users were treated with nudges and the other without that treatment.
Therefore, these data were impossible to compare. Take note. GFT was an
internal project, and yet the GFT team missed changes that their Google
search colleagues introduced. Researchers using big data must stay aware
of how their data may have drifted.

\emph{Overcoming biases in big data}

These biases have no general cure, but adopting good practices can
ensure you catch many problems before they become threats to the
validity of your research design. First, get to know your data. If you
collect your data from an API, read its documentation carefully. If you
take it from a website, then understand everything you can about how the
data on the site are maintained and curated--who puts it up, who can
remove it, how it is updated, etc. Many organizations that produce and
provide big data communicate about it, for example on company blogs or
through press releases. Pay attention to these sources of information
for things that can affect your research design. When in doubt, ask
clarifying questions. Many system administrators or database managers
are happy to help courteous researchers.

Second, try to write down your assumptions explicitly and, whenever you
can, directly test them like hypotheses \citep[see,][]{LBCC16}. For
example, GFT could have tested for algorithmic confounding by examining
whether feedback from the search engine to the user was associated with
increases in symptom-related search terms. Social media analyses should
test the assumption that their data are generated by humans by checking,
for example, that individuals in the data do not post too repetitively,
do not post inhumanly fast, and exhibit use patterns that are within
some range of normal behavior for the platform.

Finally, develop a habit of running sanity checks, even when you do not
anticipate problems and cannot formulate an explicit assumption. Since
big data is, well, big, it is not realistic to look at the numbers or
text in your data and expect to see anything useful. We do this by
drawing pictures. Make histograms, scatterplots, and other plots of your
variables and be sure the patterns make sense. When you discover a
strange pattern, investigate it until you understand it. You may very
well uncover hidden drift or the tell-tale patterns of dirty data caused
by bots or other automated interference.

\emph{Levels of analysis and measurement}

\hypertarget{measurement-stuff}{%
\subsection{measurement stuff}\label{measurement-stuff}}

\begin{itemize}
\tightlist
\item
  role of big data in the research design

  \begin{itemize}
  \tightlist
  \item
    uncommon to directly analyze
  \item
    more often: measurement
  \end{itemize}
\end{itemize}

\hypertarget{behandling-af-big-data}{%
\section{Behandling af big data}\label{behandling-af-big-data}}

Big data præsenterer både computationelle og analytiske udfordringer -
vi fokuserer på analytiske

Fællestræk: højdimensionalitet

Fælles for metoder: dimensionalitetsreduktion

Klassifikation ctr. skalering

usuperviserede tilgange

superviserede tilgange

\hypertarget{etiske-problemer-ved-big-data}{%
\section{Etiske problemer ved big
data}\label{etiske-problemer-ved-big-data}}

emotional contagion example: \citet{kramer2014experimental}

problem: lack of informed consent

complication: under surveillance capitalism, all citizens are subject to
constant experimentation w/o consent

\bibliography{references.bib}

\end{document}
