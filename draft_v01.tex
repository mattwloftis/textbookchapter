\documentclass[12pt,]{article}
\usepackage[]{mathptmx}
\usepackage{amssymb,amsmath}
\usepackage{ifxetex,ifluatex}
\usepackage{fixltx2e} % provides \textsubscript
\ifnum 0\ifxetex 1\fi\ifluatex 1\fi=0 % if pdftex
  \usepackage[T1]{fontenc}
  \usepackage[utf8]{inputenc}
\else % if luatex or xelatex
  \ifxetex
    \usepackage{mathspec}
  \else
    \usepackage{fontspec}
  \fi
  \defaultfontfeatures{Ligatures=TeX,Scale=MatchLowercase}
\fi
% use upquote if available, for straight quotes in verbatim environments
\IfFileExists{upquote.sty}{\usepackage{upquote}}{}
% use microtype if available
\IfFileExists{microtype.sty}{%
\usepackage{microtype}
\UseMicrotypeSet[protrusion]{basicmath} % disable protrusion for tt fonts
}{}
\usepackage[margin=1in]{geometry}
\usepackage{hyperref}
\hypersetup{unicode=true,
            pdftitle={`Big data' og politologisk datavidenskab},
            pdfauthor={Frederik Hjorth; Matt W. Loftis},
            pdfborder={0 0 0},
            breaklinks=true}
\urlstyle{same}  % don't use monospace font for urls
\usepackage{natbib}
\bibliographystyle{apsr}
\usepackage{graphicx}
% grffile has become a legacy package: https://ctan.org/pkg/grffile
\IfFileExists{grffile.sty}{%
\usepackage{grffile}
}{}
\makeatletter
\def\maxwidth{\ifdim\Gin@nat@width>\linewidth\linewidth\else\Gin@nat@width\fi}
\def\maxheight{\ifdim\Gin@nat@height>\textheight\textheight\else\Gin@nat@height\fi}
\makeatother
% Scale images if necessary, so that they will not overflow the page
% margins by default, and it is still possible to overwrite the defaults
% using explicit options in \includegraphics[width, height, ...]{}
\setkeys{Gin}{width=\maxwidth,height=\maxheight,keepaspectratio}
\IfFileExists{parskip.sty}{%
\usepackage{parskip}
}{% else
\setlength{\parindent}{0pt}
\setlength{\parskip}{6pt plus 2pt minus 1pt}
}
\setlength{\emergencystretch}{3em}  % prevent overfull lines
\providecommand{\tightlist}{%
  \setlength{\itemsep}{0pt}\setlength{\parskip}{0pt}}
\setcounter{secnumdepth}{0}
% Redefines (sub)paragraphs to behave more like sections
\ifx\paragraph\undefined\else
\let\oldparagraph\paragraph
\renewcommand{\paragraph}[1]{\oldparagraph{#1}\mbox{}}
\fi
\ifx\subparagraph\undefined\else
\let\oldsubparagraph\subparagraph
\renewcommand{\subparagraph}[1]{\oldsubparagraph{#1}\mbox{}}
\fi

%%% Use protect on footnotes to avoid problems with footnotes in titles
\let\rmarkdownfootnote\footnote%
\def\footnote{\protect\rmarkdownfootnote}

%%% Change title format to be more compact
\usepackage{titling}

% Create subtitle command for use in maketitle
\providecommand{\subtitle}[1]{
  \posttitle{
    \begin{center}\large#1\end{center}
    }
}

\setlength{\droptitle}{-2em}

  \title{`Big data' og politologisk datavidenskab}
    \pretitle{\vspace{\droptitle}\centering\huge}
  \posttitle{\par}
  \subtitle{Udkast, januar 2020}
  \author{Frederik Hjorth \\ Matt W. Loftis}
    \preauthor{\centering\large\emph}
  \postauthor{\par}
    \date{}
    \predate{}\postdate{}
  

\begin{document}
\maketitle

Draft motivation:

Training, job market demands, sexy research, etc. increasingly value the
ability to work with big data. It's not always clear what one means by
big data or how one works with it, but there is a consensus that doing
so requires skills. We agree, and we present some of these skills here,
along with a description of what big data means and in what ways we work
with it.

cites: \citet{mullainathan2017machine}, \citet{varian2014big}

\hypertarget{what-is-big-data}{%
\subsection{What is big data?}\label{what-is-big-data}}

\begin{itemize}
\tightlist
\item
  Operational definition

  \begin{itemize}
  \tightlist
  \item
    e.g.~``Volume, Variety, and Velocity''; Harford (re: ``found data'')
  \item
    `digital exhaust' / metadata vs.~purpose-built systems
  \end{itemize}
\item
  How/why big data became what it is in the zeitgeist

  \begin{itemize}
  \tightlist
  \item
    Classic advantages: Big / always on
  \item
    Classic disadvantages: metered/restricted access + expertise barrier
  \end{itemize}
\end{itemize}

\hypertarget{first-skill-of-working-wbig-data-research-design}{%
\subsection{\texorpdfstring{First skill of working w/big data:
\emph{Research
Design}}{First skill of working w/big data: Research Design}}\label{first-skill-of-working-wbig-data-research-design}}

Systems that collect big data are purpose-built, and the purpose is
never political science research. This holds true even for the most
research-friendly data sources.

\begin{itemize}
\tightlist
\item
  Big data typically \emph{not designed for research}, i.e.

  \begin{itemize}
  \tightlist
  \item
    it's dirty (re: Salganik) -- human behavior is mixed together with
    actions taken by bots/automated systems
  \item
    drifting -- Needs of \emph{actual} system maintainers / users may be
    completely orthogonal to needs of political science research (or
    even opposed, consider FB)
  \item
    algorithmically confounded -- large-scale systems have robot
    nannies. These include everything from spell checkers to YouTube's
    recommendation algorithm. The observed behavior we find in big data
    is a consequence of the (generally) unobservable interaction between
    humans and these algorithms.
  \end{itemize}
\item
  Standard sampling issues (nonrepresentative, systematic sampling bias)
\end{itemize}

\hypertarget{what-to-do-with-it}{%
\subsection{What to do with it:}\label{what-to-do-with-it}}

\begin{itemize}
\item
  Pattern discovery (dimensionality reduction -- a la argument in Lowe
  2013WP)

  \begin{itemize}
  \tightlist
  \item
    Generic data: Clustering, IRT, etc.
  \item
    Text: Topic modeling, text scaling, dictionaries
  \item
    Fundamental unity of goals and approach
  \item
    Variety in methods results from variation in:

    \begin{itemize}
    \tightlist
    \item
      Assumptions re: underlying model/geometry of latent space
    \item
      Related to above: something like, ``format'' of output
    \item
      Structure/nature of input data
    \item
      Amount of domain expertise applied to structure results
    \item
      Assumptions about what is correlated with what
    \item
      Level of computational intensity
    \end{itemize}
  \end{itemize}
\item
  Classification / prediction
\item
  Explanatory modeling
\end{itemize}

\bibliography{references.bib}


\end{document}
