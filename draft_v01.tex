\documentclass[12pt,]{article}
\usepackage[]{cochineal}
\usepackage{amssymb,amsmath}
\usepackage{ifxetex,ifluatex}
\usepackage{fixltx2e} % provides \textsubscript
\ifnum 0\ifxetex 1\fi\ifluatex 1\fi=0 % if pdftex
  \usepackage[T1]{fontenc}
  \usepackage[utf8]{inputenc}
\else % if luatex or xelatex
  \ifxetex
    \usepackage{mathspec}
  \else
    \usepackage{fontspec}
  \fi
  \defaultfontfeatures{Ligatures=TeX,Scale=MatchLowercase}
\fi
% use upquote if available, for straight quotes in verbatim environments
\IfFileExists{upquote.sty}{\usepackage{upquote}}{}
% use microtype if available
\IfFileExists{microtype.sty}{%
\usepackage{microtype}
\UseMicrotypeSet[protrusion]{basicmath} % disable protrusion for tt fonts
}{}
\usepackage[margin=1in]{geometry}
\usepackage{hyperref}
\hypersetup{unicode=true,
            pdftitle={`Big data' og politologisk datavidenskab},
            pdfauthor={Frederik Hjorth; Matt W. Loftis},
            pdfborder={0 0 0},
            breaklinks=true}
\urlstyle{same}  % don't use monospace font for urls
\usepackage{natbib}
\bibliographystyle{apsr}
\usepackage{graphicx,grffile}
\makeatletter
\def\maxwidth{\ifdim\Gin@nat@width>\linewidth\linewidth\else\Gin@nat@width\fi}
\def\maxheight{\ifdim\Gin@nat@height>\textheight\textheight\else\Gin@nat@height\fi}
\makeatother
% Scale images if necessary, so that they will not overflow the page
% margins by default, and it is still possible to overwrite the defaults
% using explicit options in \includegraphics[width, height, ...]{}
\setkeys{Gin}{width=\maxwidth,height=\maxheight,keepaspectratio}
\IfFileExists{parskip.sty}{%
\usepackage{parskip}
}{% else
\setlength{\parindent}{0pt}
\setlength{\parskip}{6pt plus 2pt minus 1pt}
}
\setlength{\emergencystretch}{3em}  % prevent overfull lines
\providecommand{\tightlist}{%
  \setlength{\itemsep}{0pt}\setlength{\parskip}{0pt}}
\setcounter{secnumdepth}{5}
% Redefines (sub)paragraphs to behave more like sections
\ifx\paragraph\undefined\else
\let\oldparagraph\paragraph
\renewcommand{\paragraph}[1]{\oldparagraph{#1}\mbox{}}
\fi
\ifx\subparagraph\undefined\else
\let\oldsubparagraph\subparagraph
\renewcommand{\subparagraph}[1]{\oldsubparagraph{#1}\mbox{}}
\fi

%%% Use protect on footnotes to avoid problems with footnotes in titles
\let\rmarkdownfootnote\footnote%
\def\footnote{\protect\rmarkdownfootnote}

%%% Change title format to be more compact
\usepackage{titling}

% Create subtitle command for use in maketitle
\providecommand{\subtitle}[1]{
  \posttitle{
    \begin{center}\large#1\end{center}
    }
}

\setlength{\droptitle}{-2em}

  \title{`Big data' og politologisk datavidenskab}
    \pretitle{\vspace{\droptitle}\centering\huge}
  \posttitle{\par}
  \subtitle{Udkast, april 2020}
  \author{Frederik Hjorth \\ Matt W. Loftis}
    \preauthor{\centering\large\emph}
  \postauthor{\par}
    \date{}
    \predate{}\postdate{}
  

\begin{document}
\maketitle

`Big data' er overalt. Det gælder i dobbelt forstand: takket være
drastiske stigninger i computeres hukommelse og regnekraft indeholder
næsten alle computere i dag store, ustrukturerede datamængder. Mange af
disse data er biprodukter af menneskelig adfærd, som i dag registreres
og kvantificeres i historisk uset omfang. Men big data er også overalt i
den forstand at begrebet `big data' og beslægtede begreber er blevet
almindeligt kendte og bredt anvendte, og ikke mindst genstand for stor
kommerciel interesse. En hyppigt citeret artikel fra \emph{Harvard
Business Review} kaldte således ``data scientist'' for ``det 21.
århundredes mest sexede job'' \citep{davenport2012data}.

Alene den begrebslige udbredelse af big data gør det relevant at vide
hvad det nærmere dækker over. Men big data er også reelt et væsentligt
nybrud i forhold til de data og metoder, politologi og samfundsvidenskab
traditionelt har betjent sig af. Big data muliggør analyser af
politologiske emner som ville have været umulige med traditionelle
metoder, men kræver også nye teknikker og metodiske værktøjer.

Formålet med dette kapitel er at introducere til de datatyper og
metoder, begrebet big data dækker over. Først opridser vi begrebets
beytdning og historie. Dernæst diskuterer vi hvordan en række
karakteristika ved big data skaber særlige udfordringer i forhold til at
udvikle stærke forskningsdesigns. Herefter præsenterer vi en række
specifikke tekniske værktøjer til behandling af big data. Afslutningsvis
opridser vi nogle væsentlige etiske problematikker i relation til brugen
af big data.

cites: \citet{mullainathan2017machine}, \citet{varian2014big}

\hypertarget{hvad-er-big-data}{%
\subsection{Hvad er big data?}\label{hvad-er-big-data}}

Big data has been credited with enabling an emerging field of
computational social science with ``the capacity to collect and analyze
data with an unprecedented breadth and depth and scale''
\citep{lazeretal}. Yet, the concept of big data lives two lives. On the
one hand the popular definition is associated with exciting and
futuristic promises of data-driven science and technology. On the other
hand is what we might call the operational definition of how massive
data sets are collected, stored, and used currently by social
scientists. To unpack the role of big data in political science
research, we first consider the second aspect of the concept and then
circle back to examine what to believe about the first aspect.

One widely familiar operational definition identifies big data with the
so-called three Vs: Volume, Variety, and Velocity \citep{laney01}. Big
data, here, refers to fast-evolving and very large data sets--including
even data too large to store on a desktop computer--that contain lots of
variation. Data sources of this nature are often interesting to social
science: search engine logs, social media activity, government
administrative records, mobile telephone records, or even data stored by
passive monitoring conducted by digital devices in the physical world
have been mobilized for social science research \citep{salganik17}.
Accessing these data requires partnerships with their owners--phone
companies, governments, technology companies, etc. This can mean writing
software to access websites or remote databases or can involve more
formalized partnerships to share information securely \citep{EL14}.

The hurdle of accessing big data underscores a basic difference between
it and traditional social science data: traditional social science data
were collected for the purpose of doing social science. Social
scientists must always consider the sources and the nature of our data,
and big data has sometimes been referred to as ``found data'' or
``digital exhaust'' \citep{harford14}. What does this mean? Virtually
all big data are purpose-built for goals \emph{other than} social
research. Metadata like timestamps, follower counts, or activity
frequencies on social media web sites are not stored for science or,
necessarily, according to scientific standards. Although they may be
useful, their scientific value is an unintended byproduct (i.e.~exhaust)
of business or government activity. As such, when applying big data to
social science research we must always probe the implications of the
data's purpose for our scientific applications.\footnote{As
  \citet{salganik17} puts it, the challenges and opportunities created
  by big data follow from asking why the data were collected.}

\begin{itemize}
\tightlist
\item
  How/why big data became what it is in the zeitgeist

  \begin{itemize}
  \tightlist
  \item
    Classic advantages: Big / always on
  \item
    Classic disadvantages: metered/restricted access + expertise barrier
  \end{itemize}
\end{itemize}

\hypertarget{big-data-og-forskningsdesign}{%
\subsection{Big data og
forskningsdesign}\label{big-data-og-forskningsdesign}}

Systems that collect big data are purpose-built, and the purpose is
never political science research. This holds true even for the most
research-friendly data sources.

\begin{itemize}
\tightlist
\item
  Big data typically \emph{not designed for research}, i.e.

  \begin{itemize}
  \tightlist
  \item
    it's dirty (re: Salganik) -- human behavior is mixed together with
    actions taken by bots/automated systems
  \item
    drifting -- Needs of \emph{actual} system maintainers / users may be
    completely orthogonal to needs of political science research (or
    even opposed, consider FB)
  \item
    algorithmically confounded -- large-scale systems have robot
    nannies. These include everything from spell checkers to YouTube's
    recommendation algorithm. The observed behavior we find in big data
    is a consequence of the (generally) unobservable interaction between
    humans and these algorithms.
  \end{itemize}
\item
  Standard sampling issues (nonrepresentative, systematic sampling bias)
\end{itemize}

\hypertarget{behandling-af-big-data}{%
\subsection{Behandling af big data}\label{behandling-af-big-data}}

\begin{itemize}
\item
  Pattern discovery (dimensionality reduction -- a la argument in Lowe
  2013WP)

  \begin{itemize}
  \tightlist
  \item
    Generic data: Clustering, IRT, etc.
  \item
    Text: Topic modeling, text scaling, dictionaries
  \item
    Fundamental unity of goals and approach
  \item
    Variety in methods results from variation in:

    \begin{itemize}
    \tightlist
    \item
      Assumptions re: underlying model/geometry of latent space
    \item
      Related to above: something like, ``format'' of output
    \item
      Structure/nature of input data
    \item
      Amount of domain expertise applied to structure results
    \item
      Assumptions about what is correlated with what
    \item
      Level of computational intensity
    \end{itemize}
  \end{itemize}
\item
  Classification / prediction
\item
  Explanatory modeling
\end{itemize}

\hypertarget{etiske-problemer-ved-big-data}{%
\subsection{Etiske problemer ved big
data}\label{etiske-problemer-ved-big-data}}

\bibliography{references.bib}


\end{document}
