\PassOptionsToPackage{unicode=true}{hyperref} % options for packages loaded elsewhere
\PassOptionsToPackage{hyphens}{url}
%
\documentclass[12pt,]{article}
\usepackage[]{cochineal}
\usepackage{amssymb,amsmath}
\usepackage{ifxetex,ifluatex}
\usepackage{fixltx2e} % provides \textsubscript
\ifnum 0\ifxetex 1\fi\ifluatex 1\fi=0 % if pdftex
  \usepackage[T1]{fontenc}
  \usepackage[utf8]{inputenc}
  \usepackage{textcomp} % provides euro and other symbols
\else % if luatex or xelatex
  \usepackage{unicode-math}
  \defaultfontfeatures{Ligatures=TeX,Scale=MatchLowercase}
\fi
% use upquote if available, for straight quotes in verbatim environments
\IfFileExists{upquote.sty}{\usepackage{upquote}}{}
% use microtype if available
\IfFileExists{microtype.sty}{%
\usepackage[]{microtype}
\UseMicrotypeSet[protrusion]{basicmath} % disable protrusion for tt fonts
}{}
\IfFileExists{parskip.sty}{%
\usepackage{parskip}
}{% else
\setlength{\parindent}{0pt}
\setlength{\parskip}{6pt plus 2pt minus 1pt}
}
\usepackage{hyperref}
\hypersetup{
            pdftitle={`Big data' og politologisk datavidenskab},
            pdfauthor={Frederik Hjorth; Matt W. Loftis},
            pdfborder={0 0 0},
            breaklinks=true}
\urlstyle{same}  % don't use monospace font for urls
\usepackage[margin=1in]{geometry}
\usepackage{graphicx,grffile}
\makeatletter
\def\maxwidth{\ifdim\Gin@nat@width>\linewidth\linewidth\else\Gin@nat@width\fi}
\def\maxheight{\ifdim\Gin@nat@height>\textheight\textheight\else\Gin@nat@height\fi}
\makeatother
% Scale images if necessary, so that they will not overflow the page
% margins by default, and it is still possible to overwrite the defaults
% using explicit options in \includegraphics[width, height, ...]{}
\setkeys{Gin}{width=\maxwidth,height=\maxheight,keepaspectratio}
\setlength{\emergencystretch}{3em}  % prevent overfull lines
\providecommand{\tightlist}{%
  \setlength{\itemsep}{0pt}\setlength{\parskip}{0pt}}
\setcounter{secnumdepth}{5}
% Redefines (sub)paragraphs to behave more like sections
\ifx\paragraph\undefined\else
\let\oldparagraph\paragraph
\renewcommand{\paragraph}[1]{\oldparagraph{#1}\mbox{}}
\fi
\ifx\subparagraph\undefined\else
\let\oldsubparagraph\subparagraph
\renewcommand{\subparagraph}[1]{\oldsubparagraph{#1}\mbox{}}
\fi

% set default figure placement to htbp
\makeatletter
\def\fps@figure{htbp}
\makeatother

\usepackage{etoolbox}
\makeatletter
\providecommand{\subtitle}[1]{% add subtitle to \maketitle
  \apptocmd{\@title}{\par {\large #1 \par}}{}{}
}
\makeatother
\usepackage[]{natbib}
\bibliographystyle{apsr}

\title{`Big data' og politologisk datavidenskab}
\providecommand{\subtitle}[1]{}
\subtitle{Udkast, april 2020}
\author{Frederik Hjorth \and Matt W. Loftis}
\date{}

\begin{document}
\maketitle

`Big data' er overalt. Det gælder i dobbelt forstand: takket være
drastiske stigninger i computeres hukommelse og regnekraft indeholder
næsten alle computere i dag store, ustrukturerede datamængder. Mange af
disse data er biprodukter af menneskelig adfærd, som i dag registreres
og kvantificeres i historisk uset omfang. Men big data er også overalt i
den forstand at begrebet `big data' og beslægtede begreber er blevet
almindeligt kendte og bredt anvendte, og ikke mindst genstand for stor
kommerciel interesse. En hyppigt citeret artikel fra \emph{Harvard
Business Review} kaldte således ``data scientist'' for ``det 21.
århundredes mest sexede job'' \citep{davenport2012data}.

Alene den begrebslige udbredelse af big data gør det relevant at vide
hvad det nærmere dækker over. Men big data er også reelt et væsentligt
nybrud i forhold til de data og metoder, politologi og samfundsvidenskab
traditionelt har betjent sig af. Big data muliggør analyser af
politologiske emner som ville have været umulige med traditionelle
metoder, men kræver også nye teknikker og metodiske værktøjer.

Formålet med dette kapitel er at introducere til de datatyper og
metoder, begrebet big data dækker over. Først opridser vi begrebets
betydning og historie. Dernæst diskuterer vi hvordan en række
karakteristika ved big data skaber særlige udfordringer i forhold til at
udvikle stærke forskningsdesigns. Herefter præsenterer vi en række
specifikke tekniske værktøjer til behandling af big data. Afslutningsvis
opridser vi nogle væsentlige etiske problematikker i relation til brugen
af big data.

cites: \citet{mullainathan2017machine}, \citet{varian2014big}

\hypertarget{hvad-er-big-data}{%
\section{Hvad er big data?}\label{hvad-er-big-data}}

I en toneangivende artikel peger \citet{lazeretal} på big data som
kilden til en ny type samfundsvidenskab, ``computational social
science'', med ``kapacitet til at indsamle og analysere data med
historisk uset bredde, dybde og omfang''. Men begrebet big data lever to
liv. På den ene side er den populære definition af begrebet forbundet
med futuristiske løfter om ny, datadrevet videnskab og teknologi. På den
anden side står hvad man kunne kalde den operationelle definition af
hvordan store datamænger indsamles, lagres og analyseres af
samfundsvidenskabsfolk. For at belyse betydningen af big data for
politologi betragter vi først denne anden, operationelle betydning af
begrebet inden vi vender tilbage til den første, populære betydning.

En bredt anvendt operationel definition identificerer big data med de
såkaldte `tre V'er': \emph{Volume}, \emph{Variety} og \emph{Velocity},
dvs. omfang, variation og hastighed \citep{laney01}. Big data refererer
her til foranderlige og meget store datasæt -- inklusive data der er for
store til at gemme på en almindelig pc -- der indeholder masser af
variation. Datakilder af denne art er ofte interessante for
samfundsvidenskab: søgemaskinelogfiler, sociale medieaktiviteter,
offentlige registre, mobiltelefonregistre eller endda data gemt ved
passiv overvågning udført af digitale enheder i den fysiske verden er
alle blevet anvendt til samfundsvidenskabelig forskning
\citep{salganik17}. Adgang til disse data kræver partnerskaber med deres
ejere -- telefonfirmaer, regeringer, teknologiselskaber osv. Dette kan
indebære at man skriver software til at få adgang til websteder eller
eksterne databaser eller kan involvere mere formaliserede partnerskaber
for at dele information sikkert \citep{EL14}. I afsnittet om
``Anskaffelse af data'' nedenfor beskriver vi nærmere hvordan det kan
finde sted.

Dette peger også på en vigtig forskel mellem big data og traditionelle
samfundsvidenskabelige data: traditionelle samfundsvidenskabelige data
er typisk indsamlet med samfundsvidenskab som formål. I modsætning
hertil omtales big data til tider som ``fundne data'' eller ``digital
udstødning'' \citep{harford14}. Hvad betyder det? Næsten alle big data
er udviklet til \emph{andre formål} end samfundsforskning. Metadata
såsom tidsmarkører, antal følgere, eller aktivitetsmål på sociale medier
lagres ikke for videnskabens skyld - eller nødvendigvis i
overensstemmelse med videnskabelige standarder.

Selv offentlige datakilder kan udvise dette problem. Betragt for
eksempel et lille, men illustrativt eksempel: lov nr. 1049 af
11/12/1996. Loven er ganske kort, færre end 20 ord. Det eneste den gør
er at ophæve lov om skoleskibsafgift. Civilstyrelsens database med al
dansk lovgivning, Retsinformation, angiver adskillige stykker metadata
om lov nr. 1049 (jf.
\texttt{https://www.retsinformation.dk/Forms/R0710.aspx?id=83509}).
Metadata angiver f.eks. den lov der ophæves, adskillige relaterede
dokumenter, lovens offentliggørelsesdato, og dens ministerområde. Men
dette sidste datapunkt er lidt forvirrende. Ministerområdet for lov nr.
1049 er angivet som ``Uddannelses- og Forskningsministeriet''. Men loven
er underskrevet af Mimi Jakobsen, som var erhvervsminister i regeringen
Poul Nyrup Rasmussen II. Så hvorfor er loven ikke tilknyttet
Erhvervsministeriet? Fejlen opstår fordi Civilstyrelsen opdaterer
lovgivning løbende så den afspejler lovgivningens ressortområde \emph{i
dag}. Koblingen af lov nr. 1049 til Uddannelses- og
Forskningsministeriet er indlysende forkert hvis du skal bruge
historiske data om dansk lovgivning. Uheldigvis for politologer er det
korrekt hvis du - som Civilstyrelsen - ikke har til formål at bedrive
historisk forskning, men i stedet vil organisere gældende dansk
lovgivning efter ministerområder. For at gøre ondt værre kan
Retsinformation ændre sig yderligere i fremtiden på måder der ikke
gavner politologien, alt efter hvad der tjener Civilstyrelsens behov.

Når man analyserer big data der er produceret til et eksisterende privat
eller offentligt formål lurer problemer af denne type konstant. Selvom
big data kan være enormt værdifulde for samfundsvidenskaben er deres
værdi en \emph{utilsigtet bivirkning} af kommerciel aktivitet eller
myndighedsudøvelse. Når vi anvender big data i forskningsøjemed er det
derfor altid vigtigt at forstå hvorfor og hvordan data er opstået til at
begynde med, og tænke igennem hvilke implikationer det har for vores
videnskabelige anvendelse. Som \citet{salganik17} formulerer det følger
både udfordringer og muligheder ved big data af at spørge sig selv
hvorfor data blev indsamlet i første omgang.

\hypertarget{kilder-til-big-data}{%
\section{Kilder til big data}\label{kilder-til-big-data}}

Det er ofte en møjsommelig proces at indsamle samfundsvidenskabelige
data. Derfor fremhæves det ofte som en fordel ved big data at
undersøgelsens subjekter selv genererer data: en forsker kan eksempelvis
indsamle millioner af tweets om et politisk emne uden skulle uddele et
eneste spørgeskema. Men selv om data er genereret på forhånd er det ikke
ligetil at \emph{anskaffe} sig data. Der er groft sagt tre måder man kan
gøre det på.

Den første og mest umiddelbare måde er at udtrække data direkte fra
websider, typisk kaldet \emph{scraping}. Scraping udnytter at indholdet
på de fleste større websider kommer fra databaser som fremstiller
indholdet i websider med en konsistent struktur. Ved at hente kildekoden
til disse websider, på samme måde som en webbrowser gør det, kan man
udtrække data på en konsistent måde. Hvis man f.eks. besøger websiden
for \emph{Lov om ophævelse af lov om skoleskibsafgift} hos
Retsinformation finder man i sidens kildekode bl.a. dette:

\begin{verbatim}
<div class="metadata-summary">
            <span class="kortNavn">LOV nr 1049 af 11/12/1996 Gældende</span><br>
      <div class="ressort">
                Offentliggørelsesdato: 12-12-1996<br>
        Uddannelses- og Forskningsministeriet
            </div>
        </div>
\end{verbatim}

Kodestumpen viser at Retsinformations database lagrer lovens navn i
feltet \texttt{kortNavn} og lovens offentliggørelsesdato og
ressortområde i feltet \texttt{ressort}. Takket være den stringente
kodestruktur er det nemt at gemme disse og andre metadata i et
analyserbart format. Og fordi kodestrukturen er ens på tværs af love hos
Retsinformation kan man scrape data om tusindvis af andre love med samme
lille stykke kode.

Når man indsamler data ved hjælp af scraping tilgår man i princippet
data på samme måde som en almindelig internetbruger der benytter sig af
en browser. Men fordi scraping gør det muligt at hente kolossale
datamængder er det også en kontroversiel praksis. Et illustrativt
eksempel på det kommer fra en meget omtalt juridisk strid mellem det
sociale netværk LinkedIn og analysefirmaet HiQ. En del af HiQ's
forretningsmodel er at analysere arbejdsmarkedet for it-specialister, og
HiQ har bl.a. høstet data ved at scrape offentlige profiler fra
LinkedIn. I 2017 sagsøgte LinkedIn HiQ med påstand om at HiQ's
scraping-praksis var et brud på amerikansk it-lovgivning. HiQ fik til
sidst medhold i at virksomheden kunne scrape data fra offentlige
LinkedIn-sider uden tilsagn fra LinkedIn, men sagen illustrerer at
scraping ofte finder sted i en juridisk gråzone.

Kodestumpen om \emph{Lov om ophævelse af lov om skoleskibsafgift} kommer
fra Retsinformation, og det er som hovedregel ikke forbudt at scrape
data fra offentlige hjemmesider, så længe man ikke urimeligt belaster
udbyderens servere. Man bør dog uanset kilden altid sikre sig tilsagn
fra dataudbyderen før man går i gang med at scrape data.

En anden måde at hente data på er gennem såkaldte API'er. API står for
\emph{Application Programming Interface} og er en slags kontrolleret
adgang til data hos en dataudbyder. API'er indebærer altså ikke samme
juridiske usikkerheder som scraping, da udbyderen selv stiller data til
rådighed og definerer rammerne herfor. Eksempelvis har mange API'er
\emph{rate limits} der sætter grænser for hvor meget data man kan hente
ad gangen.

Princippet om at big data ikke er lavet for samfundsforskningens skyld
gælder også for API'er. Det egentlige formål for de fleste API'er er at
dele data på tværs af kommercielle platforme. For eksempel er det API'er
der muliggør at et online-medie kan vise hvilke af ens egne
Facebook-venner der har `liket' en specifik artikel, fordi avisen kan
tilgå data om læserens Facebook-netværk gennem Facebooks API. Men mange
sociale netværk stiller meget righoldige data til rådighed for forskere
gennem API'er. For eksempel bruger \citet{hjorth2019}, som studerer
rækkevidden af online misinformation, Twitters API til at indsamle data
om ca. 13 millioner følgere af ca. 10.000 Twitter-konti. Offentlige
myndigheder stiller også i stigende grad data til rådighed gennem
API'er. Eksempelvis stiller Folketinget data om medlemmer, forhandlinger
og lovarbejde til rådighed gennem en API.

En tredje måde at få adgang til big data er gennem et egentligt
samarbejde med virksomheder der lagrer big data. For eksempel
rapporterer \citet{bond2012} om et eksperiment, hvor samfundsforskere i
samarbejde med Facebook randomiserede hvilken type information
Facebook-brugere fik om deres venners stemmeadfærd. I kraft af
samarbejdet kunne forskerne udføre eksperimentet i en uhørt stor skala:
eksperimentet involverede i alt 61 millioner Facebook-brugere.

Studiet af Bond m.fl. er exceptionelt fordi det kombinerer kvaliteterne
ved big data og eksperimentel metode. Mange forskere gør derfor også en
stor indsats for at etablere samarbejder med virksomheder og
organisationer der kan give dem adgang til data der ellers ville være
utilgængelige. Men samarbejde med virksomheder om big data er ikke uden
faldgruber. For det første kræver det ofte et betydeligt bureaukratisk
benarbejde at etablere et samarbejde. For det andet, og mere principielt
problematisk, er virksomheder og organisationer sjældent interesserede i
forskning der stiller dem selv i et dårligt lys. Det kan betyde at nogle
typer undersøgelser prioriteres på bekostningen af andre alene fordi de
passer til store teknologivirksomheders dagsordener. Eksempelvis
konkluderede \citet{bond2012} at Facebook-kampagnen havde en gunstig
effekt på valgdeltagelse. Men det er uklart om forskerne havde haft
samme frihedsgrader til at studere de negative konsekvenser af at bruge
Facebook.

\hypertarget{big-data-og-forskningsdesign}{%
\section{Big data og
forskningsdesign}\label{big-data-og-forskningsdesign}}

Big data in political science research designs: - addressing confounding
- it's dirty (re: Salganik) -- human behavior is mixed together with
actions taken by bots/automated systems - drifting -- Needs of
\emph{actual} system maintainers / users may be completely orthogonal to
needs of political science research (or even opposed, consider FB) -
algorithmically confounded -- large-scale systems have robot nannies.
These include everything from spell checkers to YouTube's recommendation
algorithm. The observed behavior we find in big data is a consequence of
the (generally) unobservable interaction between humans and these
algorithms. - Standard sampling issues (nonrepresentative, systematic
sampling bias) - its role - uncommon to directly analyze - more often:
measurement

\hypertarget{behandling-af-big-data}{%
\section{Behandling af big data}\label{behandling-af-big-data}}

\hypertarget{anskaffelse-af-data}{%
\subsection{Anskaffelse af data}\label{anskaffelse-af-data}}

\begin{itemize}
\tightlist
\item
  Getting data: scraping, APIs, text databases
\end{itemize}

\hypertarget{usuperviserede-tilgange}{%
\subsection{Usuperviserede tilgange}\label{usuperviserede-tilgange}}

\hypertarget{superviserede-tilgange}{%
\subsection{Superviserede tilgange}\label{superviserede-tilgange}}

\hypertarget{tekst-som-data}{%
\subsection{Tekst som data}\label{tekst-som-data}}

\begin{itemize}
\item
  Pattern discovery (dimensionality reduction -- a la argument in Lowe
  2013WP)

  \begin{itemize}
  \tightlist
  \item
    Generic data: Clustering, IRT, etc.
  \item
    Fundamental unity of goals and approach
  \item
    Variety in methods results from variation in:

    \begin{itemize}
    \tightlist
    \item
      Assumptions re: underlying model/geometry of latent space
    \item
      Related to above: something like, ``format'' of output
    \item
      Structure/nature of input data
    \item
      Amount of domain expertise applied to structure results
    \item
      Assumptions about what is correlated with what
    \item
      Level of computational intensity
    \end{itemize}
  \end{itemize}
\item
  Text: Topic modeling, text scaling, dictionaries
\item
  Classification / prediction
\item
  Explanatory modeling
\end{itemize}

\hypertarget{etiske-problemer-ved-big-data}{%
\section{Etiske problemer ved big
data}\label{etiske-problemer-ved-big-data}}

emotional contagion example: \citet{kramer2014experimental}

problem: lack of informed consent

complication: under surveillance capitalism, all citizens are subject to
constant experimentation w/o consent

\bibliography{references.bib}

\end{document}
